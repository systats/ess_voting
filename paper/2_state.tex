\documentclass[]{article}
\usepackage{lmodern}
\usepackage{amssymb,amsmath}
\usepackage{ifxetex,ifluatex}
\usepackage{fixltx2e} % provides \textsubscript
\ifnum 0\ifxetex 1\fi\ifluatex 1\fi=0 % if pdftex
  \usepackage[T1]{fontenc}
  \usepackage[utf8]{inputenc}
\else % if luatex or xelatex
  \ifxetex
    \usepackage{mathspec}
  \else
    \usepackage{fontspec}
  \fi
  \defaultfontfeatures{Ligatures=TeX,Scale=MatchLowercase}
\fi
% use upquote if available, for straight quotes in verbatim environments
\IfFileExists{upquote.sty}{\usepackage{upquote}}{}
% use microtype if available
\IfFileExists{microtype.sty}{%
\usepackage{microtype}
\UseMicrotypeSet[protrusion]{basicmath} % disable protrusion for tt fonts
}{}
\usepackage[margin=1in]{geometry}
\usepackage{hyperref}
\hypersetup{unicode=true,
            pdfborder={0 0 0},
            breaklinks=true}
\urlstyle{same}  % don't use monospace font for urls
\usepackage{graphicx,grffile}
\makeatletter
\def\maxwidth{\ifdim\Gin@nat@width>\linewidth\linewidth\else\Gin@nat@width\fi}
\def\maxheight{\ifdim\Gin@nat@height>\textheight\textheight\else\Gin@nat@height\fi}
\makeatother
% Scale images if necessary, so that they will not overflow the page
% margins by default, and it is still possible to overwrite the defaults
% using explicit options in \includegraphics[width, height, ...]{}
\setkeys{Gin}{width=\maxwidth,height=\maxheight,keepaspectratio}
\IfFileExists{parskip.sty}{%
\usepackage{parskip}
}{% else
\setlength{\parindent}{0pt}
\setlength{\parskip}{6pt plus 2pt minus 1pt}
}
\setlength{\emergencystretch}{3em}  % prevent overfull lines
\providecommand{\tightlist}{%
  \setlength{\itemsep}{0pt}\setlength{\parskip}{0pt}}
\setcounter{secnumdepth}{0}
% Redefines (sub)paragraphs to behave more like sections
\ifx\paragraph\undefined\else
\let\oldparagraph\paragraph
\renewcommand{\paragraph}[1]{\oldparagraph{#1}\mbox{}}
\fi
\ifx\subparagraph\undefined\else
\let\oldsubparagraph\subparagraph
\renewcommand{\subparagraph}[1]{\oldsubparagraph{#1}\mbox{}}
\fi

%%% Use protect on footnotes to avoid problems with footnotes in titles
\let\rmarkdownfootnote\footnote%
\def\footnote{\protect\rmarkdownfootnote}

%%% Change title format to be more compact
\usepackage{titling}

% Create subtitle command for use in maketitle
\newcommand{\subtitle}[1]{
  \posttitle{
    \begin{center}\large#1\end{center}
    }
}

\setlength{\droptitle}{-2em}
  \title{}
  \pretitle{\vspace{\droptitle}}
  \posttitle{}
  \author{}
  \preauthor{}\postauthor{}
  \date{}
  \predate{}\postdate{}


\begin{document}

These past years, populism has received great attention from social
scientists and political commentators {[}@mudde2004populist;
@panizza2005populism{]}. The term `populism' is both widely used and
disputed {[}@roberts2006populism; @barr2009populists{]}. Several
scientists have taken on the challenge of conceptualising populism with
only a small number of widely acknowledged characteristics. Often, the
concept is broken down to political, economic, social, and discursive
features and analyzed from numerous theoretical perspectives including
democratic and modernization theory, social movement theory, party
politics and political psychology {[}@postel2007populist;
@goodliffe2012resurgence; @acemoglu2013political{]}. Within the wide
range of literature there is a general agreement that populism is
context-dependent and culture-bound, therefore strongly variable across
countries. By the means of cross-national analyses, @mudde2012populism
were able to gain extensive insight into populism and democracy in Latin
America, Canada, Eastern and Western Europe. Further scholars
differentiate populism based on historical periods with studies using
data going back to the late 19th century {[}@arter2010breakthrough;
@rosenthal2012steep; @levitsky2013resurgence{]}. Furthermore, populism
cuts across ideological cleavages {[}@kaltwasser2014responses{]}: in
Europe, an exclusionary right-wing variant of populism emerged in the
1980s---and has intensified since---targeting mostly immigrants and
national minorities {[}@mudde2007populist; @ivarsflaten2008unites;
@art2011inside; @berezin2013normalization{]}. Finding common traits
which combine various populist activites across several countries
remains a great challenge. The task therefore is to explain how specific
circumstances and culture nature populists politics and how these in
turn impact the political sphere {[}@arter2010breakthrough{]}. Despite
such difficulties, it is possible to conceptualise populism by clearly
identifing the key features of the phenomenon to be observed, allowing a
comparison of populist politics across contexts.


\end{document}
